\documentclass{article}
\usepackage{fullpage}
\usepackage[a4paper, tmargin=12pt]{geometry}
\usepackage[version=4]{mhchem}
\usepackage{amsmath}
\usepackage{tikz}
\usepackage{pgfplots}
\usepackage{float}
\parindent 0 px
\pgfplotsset{compat=1.18}

\title{Chemical Kinetics}
\author{solara}
\date{\today}

\begin{document}
\maketitle
Reaction rate:
$$\ce{A + B <=> C + D}$$
$$\ce{2H2 + O2 <=> 2H2O}$$

$$\overrightarrow{forward\ rate}$$
$$\overleftarrow{reverse\ rate}$$\\
Reaction rate is characterized by the number of reactions per second. At equilibrium, the forward rate equals the reverse rate.

$$\nu d = [A][B]$$\\
Uppercase letters represent chemical components, lowercase letters are \textsc{the stoichiometric coefficient}.
$$\nu d = k[A]^a[B]^b$$
$$\nu i = k'[C]^c[D]^d$$\\
The constant $k$ relies on the affinity between $A$ and $B$.
The constant $k'$ relies on the affinity between $C$ and $D$.

$$\nu d = \nu i$$
$$k[A]^a[B]^b = k'[C]^c[D]^d$$
$$\frac{k}{k'} = \frac{[C]^c[D]^d}{[A]^a[B]^b}$$
$$\frac{k}{k'} = K$$

The equilibrium constant $K$ is a value unique to each reaction. The higher the value of $K$, the higher the reaction rate is.

$$\ce{4Fe + 3O2 -> 2Fe2O3}$$
$$\ce{CH4 + 2O2 -> CO2 + 2H2O}$$
\end{document}
