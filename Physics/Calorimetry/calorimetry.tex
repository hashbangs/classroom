\documentclass[11pt]{article}
\usepackage{fullpage}
\usepackage[a4paper]{geometry}
\usepackage[version=4]{mhchem}
\usepackage{textcomp, gensymb}

\title{Calorimetry}
\author{solara}
\date{\today}

\begin{document}

\maketitle

Heat transfer: relation between mass, time and energy.

\vspace{1cm}
Heating a mass of water of 1 kg from 20°C to 30°C demands $1/2$ less energy than would 
a mass of 2 kg from 20°C to 30°C ; and $1/2$ less energy than needed to raise 1 kg from 20°C to 40°C. \\
We observe a double proportionality between energy, mass, \& the difference in temperature $\Delta$T.

$$Q=Cm\Delta T$$
$$Q\ heat = C\ [J/kg/\degree C] * m\ [kg] * \Delta T\ [\degree C]$$
\vspace{0cm}

$C$ is the coefficient of heat capacity. It's the quantity of heat needed to raise one kg of matter by 1\degree C.
If the value of $C$ is high, the material will need more energy to be heated. A high $C$ characterizes a high
resistance to temperature change. (thermal inertia) \\
To cool down 1 kg of water by 1\degree C, you have to substract C.

$$C_{water} = 4180\ J*kg^{-1}*\degree C^{-1}$$
$$4180\ J = 1\ kcal$$
$$C_{oil} \sim 2000\ J*kg^{-1}*\degree C^{-1}$$
$$C_{metals} \sim\ \leq 1000\ J*kg^{-1}*\degree C^{-1}$$
\vspace{0cm}

The high thermal inertia of water: \\
\ce{->} stabilizes earth's climates. Oceanic climates have a shorter amplitude than continental ones.
(like Belgium, the Nederlands and the UK) \\
\ce{->} stabilizes the temperature of the human body. $C_{human} \approx C_{water}$ .

\end{document}
